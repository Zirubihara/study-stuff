\documentclass[a4paper,11pt]{article}
\usepackage[utf8]{inputenc}
\usepackage[polish]{babel}
\usepackage{listings}
\usepackage{xcolor}
\usepackage{booktabs}
\usepackage{longtable}

% Python listing style
\lstdefinestyle{pythonstyle}{
    language=Python,
    basicstyle=\ttfamily\small,
    keywordstyle=\color{blue}\bfseries,
    commentstyle=\color{gray}\itshape,
    stringstyle=\color{red},
    numbers=left,
    numberstyle=\tiny\color{gray},
    stepnumber=1,
    numbersep=8pt,
    frame=single,
    breaklines=true,
    captionpos=b
}

\lstset{style=pythonstyle}

\begin{document}

\section{Streamlit - Implementacja Dashboard}

\subsection{Wprowadzenie}

Streamlit to framework dedykowany do tworzenia interaktywnych dashboardów.
W przeciwieństwie do pozostałych bibliotek (Plotly, Bokeh, Matplotlib, Holoviews),
które generują statyczne pliki HTML lub obrazy PNG, Streamlit tworzy pełnoprawne
aplikacje webowe działające na serwerze.

\subsection{Porównanie Lines of Code (LOC)}

Tabela~\ref{tab:loc_all_charts} przedstawia porównanie złożoności implementacji
dla wszystkich 7 wykresów.

\begin{table}[h]
\centering
\caption{Porównanie Lines of Code dla 7 wykresów}
\label{tab:loc_all_charts}
\begin{tabular}{@{}lcccccc@{}}
\toprule
\textbf{Wykres} & \textbf{Plotly} & \textbf{Holoviews} & \textbf{Streamlit} & \textbf{Matplotlib} & \textbf{Bokeh} \\
\midrule
Chart 1 (Bar)          & 8  & 12 & \textbf{15} & 20 & 25 \\
Chart 2 (Grouped)      & 10 & 15 & \textbf{18} & 25 & 35 \\
Chart 3 (Bar)          & 8  & 12 & \textbf{14} & 18 & 22 \\
Chart 4 (Line)         & 9  & 13 & \textbf{20} & 22 & 28 \\
Chart 5 (Bar)          & 8  & 12 & \textbf{16} & 19 & 24 \\
Chart 6 (Bar)          & 8  & 12 & \textbf{15} & 19 & 23 \\
Chart 7 (Bar)          & 8  & 12 & \textbf{14} & 18 & 22 \\
\midrule
\textbf{Średnia}       & 8.4 & 12.6 & \textbf{16.0} & 20.1 & 25.6 \\
\textbf{Względem Plotly} & 1.0× & 1.5× & \textbf{1.9×} & 2.4× & 3.0× \\
\bottomrule
\end{tabular}
\end{table}

\textbf{Wnioski:}
\begin{itemize}
    \item Plotly jest najkrótszy (8.4 LOC średnio)
    \item Streamlit zajmuje środkową pozycję (16.0 LOC)
    \item Bokeh jest najdłuższy (25.6 LOC)
    \item Streamlit oferuje dodatkowe funkcje UI (metryki, filtry)
\end{itemize}

\newpage
\subsection{Chart 1: Execution Time - Implementacja}

\subsubsection{Plotly Express - 8 LOC (Najkrótszy)}

\begin{lstlisting}[caption={Plotly Express - Prosty wykres słupkowy},label={lst:plotly_chart1}]
import plotly.express as px

df = pd.DataFrame({
    'Library': ['Pandas', 'Polars', 'PyArrow'],
    'Time': [11.0, 1.51, 5.31]
})

fig = px.bar(df, x='Library', y='Time', color='Library')
fig.write_html('output.html')
\end{lstlisting}

\textbf{Zalety:}
\begin{itemize}
    \item Minimalny kod (8 linii)
    \item Deklaratywne API
    \item Automatyczne kolory i tooltips
\end{itemize}

\textbf{Wady:}
\begin{itemize}
    \item Brak UI components (metrics, filters)
    \item Statyczny HTML (brak reaktywności)
\end{itemize}

\subsubsection{Streamlit - 15 LOC (Dashboard)}

\begin{lstlisting}[caption={Streamlit - Wykres z metrykami},label={lst:streamlit_chart1}]
import streamlit as st
import plotly.express as px

st.subheader("Chart 1: Execution Time Comparison")

df = pd.DataFrame({
    'Library': ['Pandas', 'Polars', 'PyArrow'],
    'Time': [11.0, 1.51, 5.31]
})

# Metryki - UNIKALNE dla Streamlit
col1, col2, col3 = st.columns(3)
col1.metric("Fastest", "Polars", "1.51s")
col2.metric("Average", "All", "5.94s")
col3.metric("Slowest", "Pandas", "11.0s")

fig = px.bar(df, x='Library', y='Time', color='Library')
st.plotly_chart(fig, use_container_width=True)
\end{lstlisting}

\textbf{Zalety:}
\begin{itemize}
    \item \texttt{st.metric()} - karty metryk z deltą
    \item \texttt{st.columns()} - elastyczny layout
    \item Automatyczna reaktywność
    \item Responsive design (use\_container\_width)
\end{itemize}

\textbf{Wady:}
\begin{itemize}
    \item Wymaga serwera (\texttt{streamlit run app.py})
    \item Dłuższy o 88\% vs Plotly
    \item Nie generuje standalone HTML
\end{itemize}

\subsection{Unikalne Cechy Streamlit}

Streamlit oferuje komponenty niedostępne w innych bibliotekach:

\begin{table}[h]
\centering
\caption{Unikalne komponenty Streamlit}
\label{tab:streamlit_unique}
\begin{tabular}{@{}lll@{}}
\toprule
\textbf{Komponent} & \textbf{Funkcja} & \textbf{Przykład użycia} \\
\midrule
\texttt{st.metric()} & Karty metryk z deltą & Fastest/Slowest/Average \\
\texttt{st.columns()} & Layout w kolumnach & 3-kolumnowy grid metryk \\
\texttt{st.tabs()} & Zakładki & Multiple charts per page \\
\texttt{st.sidebar} & Boczny panel & Nawigacja/filtry \\
\texttt{st.expander()} & Rozwijane sekcje & Dodatkowe analizy \\
\texttt{st.multiselect()} & Checkbox list & Filtrowanie bibliotek \\
\texttt{st.dataframe()} & Interaktywna tabela & Sortowanie danych \\
\texttt{@st.cache\_data} & Cachowanie & Wydajność \\
\bottomrule
\end{tabular}
\end{table}

\subsection{Porównanie Funkcjonalności}

\begin{longtable}{@{}p{3cm}p{2cm}p{2cm}p{2cm}p{2cm}p{2cm}@{}}
\caption{Porównanie funkcjonalności bibliotek} \\
\toprule
\textbf{Feature} & \textbf{Plotly} & \textbf{Holoviews} & \textbf{Streamlit} & \textbf{Matplotlib} & \textbf{Bokeh} \\
\midrule
\endfirsthead
\multicolumn{6}{c}{\tablename\ \thetable\ -- kontynuacja} \\
\toprule
\textbf{Feature} & \textbf{Plotly} & \textbf{Holoviews} & \textbf{Streamlit} & \textbf{Matplotlib} & \textbf{Bokeh} \\
\midrule
\endhead
\midrule
\multicolumn{6}{r}{Kontynuacja na następnej stronie...} \\
\endfoot
\bottomrule
\endlastfoot

Interaktywność & Tak & Tak & Tak & Nie & Tak \\
Hover tooltips & Tak & Tak & Tak & Nie & Tak \\
Zoom/Pan & Tak & Tak & Tak & Nie & Tak \\
Export HTML & Tak & Tak & Nie & Nie & Tak \\
Export PNG & Tak & Tak & Screenshot & Tak & Tak \\
\midrule
UI Components & Nie & Nie & \textbf{Tak} & Nie & Nie \\
Metrics cards & Nie & Nie & \textbf{Tak} & Nie & Nie \\
Tabs & Nie & Nie & \textbf{Tak} & Nie & Nie \\
Sidebar & Nie & Nie & \textbf{Tak} & Nie & Nie \\
Filters & Nie & Nie & \textbf{Tak} & Nie & Nie \\
\midrule
LOC (avg) & 8.4 & 12.6 & 16.0 & 20.1 & 25.6 \\
Learning curve & Łatwy & Średni & Łatwy & Średni & Trudny \\
Deployment & Prosty & Prosty & Wymaga serwera & Prosty & Prosty \\
Use case & Dashboards & Analiza & \textbf{Live demos} & Publikacje & Custom apps \\
\end{longtable}

\subsection{Kiedy Używać Streamlit?}

\textbf{Idealne dla:}
\begin{itemize}
    \item \textbf{Obrona pracy magisterskiej} - live demo z interaktywnymi filtrami
    \item \textbf{Prototypy} - szybkie tworzenie MVP
    \item \textbf{Dashboardy wewnętrzne} - monitoring/analytics
    \item \textbf{Data apps} - aplikacje dla non-technical users
\end{itemize}

\textbf{NIE używaj dla:}
\begin{itemize}
    \item \textbf{Publikacje} - brak PNG/PDF export
    \item \textbf{Dokumentacja} - wymaga serwera
    \item \textbf{Embedding} - nie da się wstawić do strony
    \item \textbf{Offline use} - potrzebny Python runtime
\end{itemize}

\subsection{Podsumowanie}

Streamlit zajmuje unikalną pozycję wśród bibliotek wizualizacyjnych:

\begin{enumerate}
    \item \textbf{Nie jest to biblioteka do wykresów} - to framework do dashboardów
    \item \textbf{Średnia złożoność kodu} - 1.9× dłuższy niż Plotly, ale 60\% krótszy niż Bokeh
    \item \textbf{Najlepsze UI components} - metrics, tabs, sidebar, filters
    \item \textbf{Optymalny dla live demos} - idealny na obronę pracy
\end{enumerate}

\textbf{Rekomendacja dla pracy magisterskiej:}

\begin{itemize}
    \item Matplotlib/Plotly - wykresy do druku (PDF)
    \item Streamlit - live demo na obronie
    \item Bokeh/Holoviews - złożone interaktywne wizualizacje
\end{itemize}

\end{document}




